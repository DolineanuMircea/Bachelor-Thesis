\documentclass[12pt, class=report, crop=false]{standalone}
\usepackage{ba_thesis}

\begin{document}

\chapter{Numerical Methods and Particle in Cell Simulations}%
\label{chap:numerical-methods}

The domain of numerical simulations for laser-plasma interaction is incredibly expansive, with many approaches one can choose from, like numerically solving the Vlasov equation or using Particle-in-cell methods, and many ways to implement and optimize them. I will restrict in this thesis to the second option.

Even so, there are two key integration schemes that sit at the core of a Particle-in-cell software, the Maxwell solver and the particle pusher. In this chapter I aim to analyze in detail the most used schemes and their propeties and eventually give a bird's eye view of the currently available PIC codes out there and the implementation choices they've made.

\section{The Relevant Equations}

Particle-in-cell codes are nowadays the most popular popular tool for simulating plasma systems. One of the best references for what they are, how they work and how reliable the results are is the book by~\cite{birdsallPlasmaPhysicsComputer1995}, which describes the main numerical methods used and some of their properties. However, Particle-in-cell codes have evolved greatly in the last two decades and new techniques and optimizations have been produced and even put in practice. Even so, with the exception of quasistatic codes, they are still involved in solving the same physical equations. As such, it is useful for anyone interested in working with such software to know and understand the principles behind.

In general, PIC codes have four main componets:
\begin{enumerate}
  \item A Maxwell solver which propagates the Maxwell equations (which are relativistic invariant by themselves) in time and space on the grid
  \begin{subequations}
    \begin{align}
    \div{\vb{E}} & = \frac{\rho}{\varepsilon_0} \\
    \div{\vb{B}} & = 0 \\
    \curl{\vb{E}} & = - \pdv{\vb{B}}{t} \\
    \curl{\vb{B}} & = \mu_0 \vb{j} + \frac{1}{c^2} \pdv{\vb{E}}{t} \,;
    \end{align}
  \end{subequations}
  \item A field gatherer that interpolates the electromagnetic field at the particle positions on the grid;
  \item A particle pusher that advances the positions and velocities of each particle under the action of the Lorentz force (which we will write in its relativistic form)
  \begin{subequations}
    \begin{align}
      \label{eq:NL}
      \dv{\vb{p_\alpha}}{t} = q_\alpha \left(\vb{E}+\frac{\vb{p_\alpha}}{m_\alpha \gamma_\alpha} \cp \vb{B} \right)\\
      \gamma_\alpha = \sqrt{1+\left(\frac{\vb{p_\alpha}}{m_\alpha c}\right)^2}\,,
    \end{align}
  \end{subequations}
  where \(\alpha\) indexes each particle;
  \item Acurrent and charge depositor which computes the current and charge densities on the grid by interpolating the particle distributions.
\end{enumerate}

The main appeal of this approach is its self-consistency. That is, the total fields used are both those that are part of the electromagnetic waves that are introduced in the system (in general laser beams) and those generated by the charged particles that compose the plasma. As such we also include the long range Coulombian interaction between particles. The short range interaction, namely collisions between particles, is by default neglected since we usually simulate rarefied plasmas, but many codes now come with additional routines that include these processes. Additional routines are now developed with the advent of the high intensity laser technologies because at the corresponding energies reached by the particles quantum electrodynamical effects become relevant. Although there are quite a few PIC codes that include QED routines, there is still a long way untill these algorithms reach the efficiency and stability that of those four main ones described above. As such, the implementation of QED effects in numerical plasma simulations is currently a hot research topic.

It is mandatory to mention that while the four steps above outline a microscopic model, PIC simulations are not completely microscopic due to technological limitations regarding computing power. Instead of working with one virtual particle for one real particle, it is customary to use macro-particles. A macro-particle represents many particles of the same species (from \(10^6\) to \(10^11\) depending on the propeties of our plasma) moving colectively. These particles are obviously not localized at a single point, but rather they have a shape function attached to them to make the derivation of currents and charge densities more consistent. For a long time the use of macro-particles was not supported by argument and was a source of criticism towards PIC methods. The defense was built only on the excuse of that the simulations give very accurate statistical results. Things are different nowadays. We can now explain (quite easily in fact) that the macro-particles themselves can be interpreted a statistical ensamble of real particles. The secret lies in the Vlasov equation.

\subsection{The Connection with the Vlasov Equation}

Let us revisit the Vlasov equation, which we derived in~\cref{sec:Vlasov}

\begin{equation}
  \pdv{\rho}{t} + \vb{f}\vdot\pdv{\rho}{\vb{p}} + \vb{v}\vdot\pdv{\rho}{\vb{r}} = 0\,,
\end{equation}
where \(\rho\) was the distribution function that describes the entire system of particles, \(\vb{p}=(\vb{p_1}, \vb{p_2},\dots )\) and \(\vb{v}=(\vb{v_1}, \vb{v_2},\dots )\), \(\vb{r}=(\vb{r_1}, \vb{r_2},\dots )\) the momenta, the velocities, and the positions of the particles, and \(\vb{f}=(\vb{f_1}, \vb{f_2},\dots )\) the forces acting on each particle.

The main argument in the following discussion is a relativistic upgrade of the one in~\cite{liuHighPowerLaserPlasmaInteraction2020}. For consistency with the equations we outlined for the PIC method, I rewrite this equation in its relativistic form and considering all forces to be of the Lorentz type

\begin{equation}
  \pdv{\rho}{t} + \sum_\alpha \left[ q_\alpha \left(\vb{E_\alpha}+\frac{\vb{p_\alpha}}{m_\alpha \gamma_\alpha} \cp \vb{B_\alpha} \right) \pdv{\rho}{\vb{p_\alpha}} +  \frac{\vb{p_\alpha}}{m_\alpha \gamma_\alpha} \pdv{\rho}{\vb{r_\alpha}}\right] =0\,,
\end{equation}
where \(\alpha\) indexes all the particles in the system and the fields \(E_\alpha\) and \(B_\alpha\) are to be computed at the position of particle \(\alpha\).

The key insight now is that imposing that the particles move under the Newton-Lorentz~\cref{eq:NL} implies having a stationary solution for the distribution function. That is, the equations

\begin{subequations}
  \label{eq:boris1}
  \begin{align}
    \dv{\vb{p_\alpha}}{t} = q_\alpha \left(\vb{E}+\frac{\vb{p_\alpha}}{m_\alpha \gamma_\alpha} \cp \vb{B} \right)\\
    \dv{\vb{r_\alpha}}{t} = \frac{\vb{p_\alpha}}{m_\alpha \gamma_\alpha}\,,
  \end{align}
\end{subequations}
reduce the Vlasov equation as follows

\begin{equation}
  \pdv{\rho}{t} + \sum_\alpha \left( \pdv{\rho}{\vb{p_\alpha}}\dv{\vb{p_\alpha}}{t} + \pdv{\rho}{\vb{r_\alpha}} \dv{\vb{r_\alpha}}{t} \right) = \dv{\rho}{t} = 0 \,.
\end{equation}

Of course this is not an equivalency. While~\cref{eq:boris1} implies that the distribution function of the entire system is stationary, the reverse is not true, unless we do a rough approximation and suppose that the total distribution can be separated in a sum of independent single particle distribution functions. By employing this latter approximation we would unavoidably neglect some intrinsic interactions that take place in our system. Nonetheless, this problem does not affect the validity of our Particle-in-cell method. The thing is that while~\cref{eq:boris1} doesn't describe all the complete stationary solution of the Vlasov equation, it still describes at least one particular stationary solution. Working with superparticles is like studying the evolution of an ensable of these solutions. Thus, by including a relevant (yet not large enough to give unreasonable simulation times) number of superparticles we obtain a statistically realistic solution. Some even call PIC a Monte-Carlo method because of this.

The reverse approach is used in numerical studies of plasma physics using Vlasov codes (VC). These approaches simply try to solve the Vlasov equation as it is in order to obtain exact solutions (exact up to numerical errors). If one has the total distribution function then every statistical piece of information about the system is known. Some basics about how this can be achieved and computational optimization can be found in~\cite{silinVlasovcodeSimulationsCollisionless}. However, directly solving such a big solution with a number of variables proportional to the real number of particles in the system takes a lot of time if we try to simulate realistically sized systems.

An approach based on the splitting of the distribution function is not completely flawed. One can try to get closer to reality by using a better approximation by expanding the total distribution in a series that contains single-particle terms, two-particle terms, and so on. While it is true that this improves the solutions greatly, the cost in computation time is equally great and much refinement would have to be done.

The presentation so far should not give the reader the impression that PIC is the supperior approach. In fact, numerical heating is a common problem of PIC codes and the stability conditions for the simulations are quite restrictive, which together with the computational limitations reduce the amount of experiments you can run. An easy to follow rought sketch of the trade off between PIC and VC and a discussion on when is one better than the other is found in~\cite{bertrandVlasovModelsLaser2005}.

\section{Numerical Methods}

\subsection{Introduction to Numerical Methods for Solving Differential Equations}

\subsection{The Boris Push}
\subsubsection{Numerical Propeties}

\subsection{The FDTD Maxwell Solver}
\subsubsection{Numerical Properties}

\subsection{FFT Based Maxwell Solvers}

\section{Particle-in-cell in practice}

\section{EPOCH}
\subsection{Software Input}
\subsection{Visualizing Results}

\end{document}
