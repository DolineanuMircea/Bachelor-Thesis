\documentclass[12pt, class=report, crop=false]{standalone}
\usepackage{ba_thesis}

\begin{document}

% \addcontentsline{toc}{chapter}{Introduction}
\chapter{Electromagnetism and Laser Profiles}%
\label{chap:electromagnetism}

\section{Classical Electrodynamics}

The main principles and laws that govern the phenomena behind lasers, plasma and their interaction are those of classical electrodynamics. As such, like many others tackling this area of research, I find that adding an overview of electrodynamics is simply mandatory. My aim when it comes to differentiating this introductory review from the millions of others out there, if at all possible, is to offer thorough calculations and explanations on some aspects where I personally felt like I wanted to see things from a clearer perspective.

\subsection{Maxwell's Equations}

The Maxwell equations are~(\cite{jacksonClassicalElectrodynamics1999}):

\begin{subequations}
  \label{eq:maxwell-general}
  \begin{align}
    \div{\vb{D}} & = \rho
    \label{eq:coulomb-law-general} \\
    \div{\vb{B}} & = 0
    \label{eq:gauss-law-general} \\
    \curl{\vb{E}} & = - \pdv{\vb{B}}{t}
    \label{eq:faraday-law-general} \\
    \curl{\vb{H}} & = \vb{j} + \pdv{\vb{D}}{t}
    \label{eq:ampere-law-general} \,.
  \end{align}
\end{subequations}

In the absence of magnetic and polarizable media, \(\vb{D}=\varepsilon_0 \vb{E}\) and \(\vb{B}=\mu_0 \vb{H}\) and the equations become:

\begin{subequations}
  \label{eq:maxwell}
  \begin{align}
    \div{\vb{E}} & = \frac{\rho}{\varepsilon_0} \label{eq:coulomb-law} \\
    \div{\vb{B}} & = 0 \label{eq:gauss-law} \\
    \curl{\vb{E}} & = - \pdv{\vb{B}}{t} \label{eq:faraday-law} \\
    \curl{\vb{B}} & = \mu_0 \vb{j} + \frac{1}{c^2} \pdv{\vb{E}}{t} \,, \label{eq:ampere-law}
  \end{align}
\end{subequations}

\par
While most readers probably have already had at least a basic introduction to the phenomena from which these equations arise and are well acquainted to how to make use of these equations, I would direct those who haven't towards the book by~\cite{fleischStudentGuideMaxwell2008}%

\par
By extracting the current density from~\cref{eq:ampere-law}, computing its divergence and then replacing the electric field term using~\cref{eq:coulomb-law} one obtains the continuity equation, which relates only the field sources to one another:

\begin{equation}
  \label{eq:continuity-equation}
  \div{\vb{j}(\vb{r}, t)} + \pdv{\rho(\vb{r}, t)}{t} = 0 \,.
\end{equation}

\par
These equations are also complemented by the Lorentz force, which describes how the fields act on the sources. The expression of the Lorentz force in the contiuous case is:

\[
  \vb{F} = \int_\mathcal{V} \dd{\vb{r}'} \left[ \rho(\vb{r}', t) \vb{E}(\vb{r}', t) +
           \frac{1}{c} \vb{j}(\vb{r}', t) \cross \vb{B}(\vb{r}', t)\right] .
\]

\subsection{The Scalar and Vector Potentials}

Since the electric~(\(\vb{E}\)) and magnetic~(\(\vb{B}\)) fields are vectors, they can be described together by a total of six quantities. The sources on the other hand can be described using only four quantities: the electric charge density~\(\rho\) and the three components of the electric current density~\(\vb{j}\). This points to the fact that there is a more convenient way to describe the fields. In finding this alternative, we shall employ the following basic results from algebra:
\begin{subequations}
  \label{id:algebra}
  \begin{align}
    \div{(\curl{\vb{v}})}=0
    \label{id:div-rot} \\
    \curl{(\div{\vb{v}})}=0
    \label{id:rot-div} \\
    \curl{(\grad{f})}=0
    \label{id:rot-grad} \,,
  \end{align}
\end{subequations}
which are valid for any vector function~\(\vb{v}\) and for any scalar function~\(f\).

\par
From~\cref{eq:gauss-law,id:div-rot} one can define the vector potential \(\vb{A}\) such that

\begin{equation}
  \label{eq:vector-potential}
  \vb{B}(\vb{r}, t) = \curl{\vb{A}(\vb{r}, t)}\,.
\end{equation}
By substituting~\eqref{eq:vector-potential} in~\eqref{eq:faraday-law} one obtains

\begin{equation}
  \label{eq:faraday-vector-potential}
  \curl(\vb{E} + \pdv{\vb{A}}{t}) = 0
\end{equation}
which together with~\cref{id:rot-grad} defines the scalar potential \(\phi\)

\begin{equation}
  \label{eq:faraday-scalar-potential}
  \grad{\phi}(\vb{r}, t) = -\vb{E}(\vb{r}, t) - \pdv{\vb{A}}{t}\,.
\end{equation}
Using this in~\cref{eq:coulomb-law}

\begin{equation}
  \label{eq:coulomb-scalar-potential}
  \laplacian{\phi} + \pdv{t}\div{\vb{A}} = - \frac{\rho}{\varepsilon_0}\,.
\end{equation}
Similarly, using~\cref{eq:faraday-scalar-potential} in~\cref{eq:ampere-law} and making use of the following vector identity

\begin{equation}
  \label{eq:curl-of-curl}
  \curl(\curl{\vb{v}}) = \grad(\div{\vb{v}}) - \laplacian{\vb{v}}\,,
\end{equation}
another equation of the potentials is obtained

\begin{equation}
  \label{eq:ampere-potentials}
  \laplacian{\vb{A}} - \frac{1}{c^2}\pdv[2]{\vb{A}}{t} =
    -\mu_0 \vb{j} + \grad(\div{\vb{A}} + \frac{1}{c^2} \pdv{\phi}{t})\,.
\end{equation}

Considering that at every step in the derivation of~\cref{eq:coulomb-scalar-potential,eq:ampere-potentials} we only imposed the Maxwell equations and basic algebraic identities, it follows that~\cref{eq:coulomb-scalar-potential,eq:ampere-potentials} and~\cref{eq:maxwell} are completely equivalent. We now have reduced the six quantities describing the fields to only four: the scalar potential~\(\phi\) and the three components of the vector potential~\(\vb{A}\). This description of the fields through the potentials is quite useful since it is easily integrated in the formalism of special relativity. One can define the electomagnetic potential 4-vector such that the scalar field is the time-like component and the vector field is the space-like component.
\par
In general, when studying the dynamics of particles in an electomagnetic field, once the potentials are computed using~\cref{eq:coulomb-scalar-potential,eq:ampere-potentials} the fields are obtained from~\cref{eq:vector-potential,eq:faraday-scalar-potential} and can be used further in the expression of the Lorentz force.

\subsection{Gauge Transformation}
By a direct application of~\cref{id:algebra} one can show that a simultaneous transformation by an arbitrary well-behaved (continuous with continuous derivatives) scalar function~\(f=f(\vb{r},t)\) of the potentials:

\begin{subequations}
  \label{gauge}
  \begin{align}
    \phi \rightarrow \phi + \pdv{f}{t} \\
    \vb{A} \rightarrow \vb{A} - \grad{f} \,,
  \end{align}
\end{subequations}
leaves the electric and magnetic field unchanged. This is actually a quite natural equivalent of the intuitive fact that any potential is defined up to a constant. In the particular case of the electromagnetic potential,~\cref{gauge} define a gauge transformation. There are two widely used gauges.

\paragraph{Lorenz gauge}

\begin{equation}
  \label{eq:lorenz-gauge}
  \div{\vb{A}} + \frac{1}{c^2} \pdv{\phi}{t}=0
\end{equation}

\par
This gauge cancels the gradient in~\cref{eq:ampere-potentials}. If one works in the usual Mikovski metric~(\cite{weinbergGravitationCosmologyPrinciples1972})

\begin{equation}
  \label{metric}
  \eta_{\mu \nu} =
    \begin{pmatrix}
      -1 & 0 & 0 & 0 \\
      0 & 1 & 0 & 0 \\
      0 & 0 & 1 & 0 \\
      0 & 0 & 0 & 1
    \end{pmatrix}
\end{equation}

the d'Alembert operator is then defined as
\[
\Box = \partial^{\mu} \partial_{\mu} = \eta^{\mu \nu} \partial_{\nu} \partial_{\mu} = \laplacian{} - \frac{1}{c^{2}} \pdv[2]{}{t} \,,
\]
where \(\mu,\,\nu=\overline{0,3}\) with 0 being the temporal index and 1, 2, 3 being the spatial indices (note: in this thesis I use Einstein's summation convention whenever there is and index repeated once up and down, \(\textit{i}.\textit{e}.\) is appears as both variant and covariant in a product). By replacing this definition in~\cref{eq:coulomb-scalar-potential,eq:ampere-potentials}, it is easy to see that both \(\vb{A}\) and \(\phi\) obey a free wave equation:

\begin{subequations}
  \begin{align}
    \Box \vb{A} = -\mu_0 \vb{j}\\
    \Box \phi = -\frac{\rho}{\varepsilon_0}  \,.
  \end{align}
\end{subequations}

\paragraph{Coulomb Gauge (sometimes found as transversal/velocity gauge)}

\begin{equation}
  \div{\vb{A}} = 0
\end{equation}

\par
Under this gauge, the potential equations~\cref{eq:coulomb-scalar-potential,eq:ampere-potentials} take the form:

\begin{subequations}
  \begin{align}
    \Box \vb{A} = -\mu_0 \vb{j} + \frac{1}{c^{2}} \grad{\pdv{\phi}{t}}\\
    \laplacian{\phi} = -\frac{\rho}{\varepsilon_0}  \,.
  \end{align}
\end{subequations}

\subsection{The Poynting Theorem}

The Poynting theorem is the form of the conservation of energy in the case of electromagnetic fields interacting with charges and currents. Since it is such an important and general result, this presentation of it will start from the more general form of the Maxwell equations~\cref{eq:maxwell-general}.
\par
In the derivation of this theorem, one usually starts from the local form of the Lorentz force~(\cite{griffithsIntroductionElectrodynamics1999}):

\begin{equation*}
  \vb{F} = \delta q \vb{E} + \delta q \vb{v} \cp \vb{B}
\end{equation*}

\par

The work done by the electric field part of the force on the volume element with charge \( \delta q \) and velocity \( \vb{v} = \dv{\vb{l}}{t} \) is

\begin{equation*}
  \dd{W_{e}} = q \dd{\vb{l}} \vdot \vb{E}
\end{equation*}
and the corresponding rate of work done is

\begin{equation*}
  \dv{W_{e}}{t} = q \vb{v} \vdot \vb{E}
\end{equation*}
while for the magnetic part we have (as expected)

\begin{equation*}
  \dd{W_{m}} = \dd{\vb{l}} \vdot \vb{F_{b}} = q \dd{\vb{l}} \vdot ( \vb{v} \cp \vb{B})
\end{equation*}

\begin{equation*}
  \dv{W_{b}}{t} = q \vb{v} \vdot ( \vb{v} \cp \vb{B}) = 0 \,.
\end{equation*}

\par
Adding these contributions and generalizing for the case of a distribution of charges and currents one obtains

\begin{equation}
  \dv{W}{t} = \int_\mathcal{V} \dd{\vb{r}} \vb{E}\vdot\vb{j}
\end{equation}

By extracting \(\vb{j}\) from~\cref{eq:ampere-law-general} and replacing in the above equation we have

\begin{equation*}
  \dv{W}{t} = \int_\mathcal{V} \dd{\vb{r}} \left[ \vb{E}\vdot (\curl{\vb{H}}) - \vb{E}\vdot \pdv{\vb{D}}{t} \right]
\end{equation*}
Employing here the vector identity here

\begin{equation}
  \label{id:grad-of-vector-product}
  \grad{(\vb{u}\cp\vb{v})} = \vb{v}\vdot(\curl{\vb{u}})-\vb{u}\vdot(\curl{\vb{v}})
\end{equation}
gives

\begin{equation*}
  \dv{W}{t} = \int_\mathcal{V} \dd{\vb{r}} \left[ \vb{H}\vdot (\curl{\vb{E}}) - \grad{(\vb{E}\cp\vb{H})} - \vb{E}\vdot \pdv{\vb{D}}{t} \right] \,.
\end{equation*}
Replacing the curl of \( \vb{E}\) using Faraday's law~(\ref{eq:faraday-law-general}) we finally obtain

\begin{equation*}
  \dv{W}{t} = - \int_\mathcal{V} \dd{\vb{r}} \left[ \grad{(\vb{E}\cp\vb{H})} + \vb{E}\vdot \pdv{\vb{D}}{t} + \vb{H}\vdot \pdv{\vb{B}}{t} \right] \,.
\end{equation*}

If we restric the discussion now only to linar media (\textit{i}.\textit{e}. \(\vb{D}=\varepsilon \vb{E}\) and \(\vb{B} = \mu \vb{H}\)) a new important quantity can be defined

\begin{equation}
  \label{def:energy-density}
  w_{em} = \frac{1}{2} (\vb{E}\vdot\vb{D}+\vb{H}\vdot\vb{B})
\end{equation}
which leads to a new way to write the expression of the rate of work done by the electromagnetic field

\begin{equation}
  \label{th:poynting-0}
  \dv{W}{t} = - \int_\mathcal{V} \dd{\vb{r}} \left[\grad{(\vb{E}\cp\vb{H})} + \pdv{w_{em}}{t}\right] \,,
\end{equation}
where the Poynting vector is

\begin{equation}
  \label{def:poynting-vector}
  \vb{S}=\vb{E}\cp\vb{H}\,.
\end{equation}

In order to complete the derivation of Poynting's theorem, an we must see how it is to be interpreted. As such, a short paranthesis concerning \( w_{em}\) is in order.

\subsubsection{Electrostatic field energy density}
For a system of N stationary point-like charged particles of charges \(q_{i}\) placed at \(\vb{r_{i}} \,,\,i=\overline{1,N}\) in a medium with permittivity \(\varepsilon\), the total potential energy of the system, when neglecting the infinite self-interaction terms, is~(\cite{jacksonClassicalElectrodynamics1999})

\begin{equation*}
  W_{e} = \frac{1}{2} \sum_{i,j=1, i\neq j}^{N} \frac{q_{i} q_{j}}{4 \pi \varepsilon \abs{\vb{r_{i}}-\vb{r_{j}}}}
\end{equation*}
or, factoring out the scalar potential \(\phi (\vb{r_{i}})\) generated by all the other particles at the position of particle i,

\begin{equation*}
  W_{e} = \frac{1}{2} \sum_{i=1}^{N} q_{i} \phi (\vb{r_{i}})
\end{equation*}
This is easily generalized in integral form

\begin{equation*}
  W_{e} = \frac{1}{2} \int_\mathcal{V} \dd{\vb{r}} \rho (\vb{r}) \phi (\vb{r}) \, ,
\end{equation*}
where we use the delta-Dirac function for pointlike particles if needed.

\par
Using the fact that the electrostatic potential is defined by \(\vb{E} = - \grad{\phi}\) and replacing this in~\cref{eq:coulomb-law-general} one obtains the poison equation

\begin{equation}
 \label{eq:poisson}
 \laplacian{\phi} = - \frac{\rho}{\varepsilon} \,.
\end{equation}
With this, the integral above becomes

\begin{equation*}
  W_{e} = \frac{\varepsilon}{2} \int_\mathcal{V} \dd{\vb{r}} \phi \laplacian{\phi} = - \frac{\varepsilon}{2} \int_\mathcal{V} \dd{\vb{r}} \phi \grad{\phi} + \frac{\varepsilon}{2} \int_\mathcal{V} \dd{\vb{r}} \abs{\grad{\phi}}^{2} \, ,
\end{equation*}
where integration by parts has been used.

\par
In order to reach the desired result, we still have to perform one more integration by parts

\begin{equation*}
  \int_\mathcal{V} \dd{\vb{r}} \phi \grad{\phi} = \frac{1}{2} \int_\mathcal{V} \dd{\vb{r}} \grad{\phi^2} = \int_{\mathcal{S}_{\mathcal{V}}} \dd{\vb{a}} \phi^{2} \, ,
\end{equation*}
where in the last step we used Gauss' theorem. Now, if we itegrate over the entire space and keep in mind that the electrostatic potential should be zero at infinity, the above integral becames null. Using again the realtion between the gradient of the potential and the electric field we get

\begin{equation}
  W_{e} = \frac{\varepsilon}{2} \int_\mathcal{V} \dd{\vb{r}} \vb{E}^{2}
\end{equation}
or, equivalently,

\begin{equation}
  W_{e} = \frac{1}{2} \int_\mathcal{V} \dd{\vb{r}} \vb{E} \vdot \vb{D} \,.
\end{equation}

\par
This leads to the definition of the energy density of the electrostatic field

\begin{equation}
  \label{def:electrostatic-en-density}
  w_{e} = \frac{1}{2} \vb{E} \vdot \vb{D} \,.
\end{equation}

\subsubsection{Magnetostatic field energy density}

This time around we start with a current loop in the case of magnetostatics (\(\div{\vb{j}}=0\)). No matter the current distribution in space, since the current density is rotational, we can always divide it in individual infinitesimal current loops. A change in the magnetic flux through such a loop is given by the integral form of Faraday's law~(\ref{eq:faraday-law-general})

\begin{equation}
  e = \oint_{\gamma} \dd{\vb{l}}\vdot\vb{E} = - \dv{\phi_{B}}{t} \,,
\end{equation}
where \(\gamma\) is the closed curve describing the loop and \(\phi_{B}\) is the magnetic flux through the loop.

\par
Since the autoinduced magnetic flux is \(\phi_{B}=LI\), where \(L\) is the inductance of the loop and \(I\) the intensity of the electric current flowing in it, the electromotive force caused by autoinduction is

\begin{equation*}
  e = -L \dv{I}{t} \,.
\end{equation*}
Thus the rate of work against the increase of the current is

\begin{equation*}
  \dv{W_{B}}{t} = - I e = LI \dv{I}{t} = \dv{t} (\frac{LI^{2}}{2})\,.
\end{equation*}
With this result we obtain the energy necessary to get a current of intensity \(I\) starting through a loop:

\begin{equation*}
  W_{B} = \frac{LI^2}{2} \,.
\end{equation*}

\par
We will now eliminate \(L\) the same way we introduced it

\begin{equation*}
  \phi_{B} = LI = \int_{\mathcal{S}_\mathcal{\gamma}} \dd{\vb{a}} \vdot \vb{B} =  \int_{\mathcal{S}_\mathcal{\gamma}} \dd{\vb{a}} \vdot (\curl{\vb{A}}) = \oint_{\gamma} \dd{\vb{l}}\vdot \vb{A} \,,
\end{equation*}
where the vector potential was introduced and Stokes' theorem  was applyed.

\begin{equation*}
  W_{B} = \frac{1}{2} I \oint_{\gamma} \dd{\vb{l}}\vdot \vb{A} = \frac{1}{2} \int_\mathcal{V} \dd{\vb{r}} \vb{j}\vdot\vb{A} \,.
\end{equation*}
Here we naturally introduced the electric current density in our calculations. It can be replaced though using~\cref{eq:ampere-law-general} (we work in the confinements of magnetostatics, so there is no time dependent electric field)

\begin{equation*}
  W_{B} = \frac{1}{2} \int_\mathcal{V} \dd{\vb{r}} \vb{A}\vdot (\curl{\vb{H}}) \,.
\end{equation*}
We employ here the identity~(\ref{id:grad-of-vector-product}) to reach

\begin{equation*}
  W_{B} = \frac{1}{2} \int_\mathcal{V} \dd{\vb{r}} \vb{H}\vdot (\curl{\vb{A}}) - \frac{1}{2} \int_\mathcal{V} \dd{\vb{r}} \div{(\vb{A}\cp\vb{H})} = \frac{1}{2} \int_\mathcal{V} \dd{\vb{r}} \vb{H}\vdot \vb{B} - \frac{1}{2} \int_{\mathcal{S}_\mathcal{V}} \dd{\vb{a}} \vdot (\vb{A}\cp\vb{H}) \,.
\end{equation*}
The same trick as in the previous subsection is appliable here. By extending the integration volume over the entire space and using the fact that the vector potential should be zero at infinity, the second integral vanishes.

\begin{equation}
  W_{B} = \frac{1}{2} \int_\mathcal{V} \dd{\vb{r}} \vb{H}\vdot \vb{B}
\end{equation}

\par
The energy density of the magnetostatic field is defined to be

\begin{equation}
  \label{def:magnetostatic-en-density}
  w_{B} = \frac{1}{2} \vb{H}\vdot\vb{B} \,.
\end{equation}

\subsubsection{Interpretation of the Poynting theorem}

We can see now that~(\ref{def:energy-density}) is simply the sum of~\cref{def:electrostatic-en-density} and~\cref{def:magnetostatic-en-density}. Summing up all the previous considerations, \(w_{em}\) holds the meaning of the energy density of the electromagnetic field itself, that is, the energy density present in space due to the presence of the electric and magnetic fields.

\par
The Poynting theorem~(\ref{th:poynting-0}) can be rewritten using Gauss' theorem in its integral form

\begin{equation}
  \label{th:poynting-integral}
  \dv{W}{t} = - \int_{\mathcal{S}_\mathcal{V}} \dd{\vb{a}} \vdot \vb{S} - \dv{t} \int_\mathcal{V} \dd{\vb{r}} w_{em}
\end{equation}

or in its differential form by eliminating the integrals

\begin{equation}
  \label{th:poynting-differential}
  \vb{E}\vdot\vb{j} = - \div{\vb{S}} - \pdv{w_{em}}{t} \,.
\end{equation}

The Poynting vector has units of \(\frac{J}{m^2 s}\) and decribes the flux of energy through a surface. From this, we can conclude that the physical meaning behind~\cref{th:poynting-integral,th:poynting-differential} is that the rate of change in time of the energy inside a volume  added with the flow of energy in and out of that volume is equal to minus the work done by the fields on the sources inside the volume.

\subsection{Momentum of a System of Fields and Field Sources}
By taking the vector product of \(\vb{D}\) with~\cref{eq:faraday-law-general} and of \(\vb{B}\) with~\cref{eq:ampere-law-general} and then adding them up the following equality can be obtained:

\begin{equation}
  \vb{D}\cp (\curl{\vb{E}}) + \vb{B}\cp (\curl{\vb{H}}) = - \vb{j}\cp \vb{B} - \pdv{t} (\vb{D}\cp\vb{B}) \,.
\end{equation}
We will restric this discussion to the case where there is no polarizable or magnetic media:

\begin{equation}
  \label{expression-1}
  \varepsilon_0 \vb{E}\cp (\curl{\vb{E}}) + \frac{1}{\mu_0}\vb{B}\cp (\curl{\vb{B}}) = - \vb{j}\cp \vb{B} - \varepsilon_0 \pdv{t} (\vb{E}\cp\vb{B}) \,.
\end{equation}
Considering that the speed of light in vacuum is \(c=\frac{1}{\sqrt{\varepsilon_0 \mu_0}}\),~\cref{expression-1} becomes

\begin{equation}
  \label{expression-1}
  \varepsilon_0 (\vb{E}\cp (\curl{\vb{E}}) + c^{2} \vb{B}\cp (\curl{\vb{B}})) = - \vb{j}\cp \vb{B} - \varepsilon_0 \pdv{t} (\vb{E}\cp\vb{B}) \,.
\end{equation}

In order to proceed, some vector algebra must be discussed. In particular, we would like to evaluate the following expression:

\begin{equation*}
  \vb{v}(\div{\vb{v}}) - \vb{v}\cp(\curl{\vb{v}})
\end{equation*}

We have

\begin{equation*}
  \vb{v}\cp(\curl{\vb{v}}) = \vb{e}^{i} \varepsilon_{ijk} v^{j} (\curl{\vb{v}})_{k} = \vb{e}^{i} \varepsilon_{ijk} v^{j} \varepsilon^{lmk} \partial_{l} v_{m} =
\end{equation*}

\begin{equation*}
  = \vb{e}^{i} (\delta_{i}^{l}\delta_{j}^{m}-\delta_{i}^{m}\delta_{j}^{l}) v^{j} \partial_{l} v_{m} =
  \vb{e}^{i} \left[ v^{j} \partial_{i} v_{j} - v^{j} \partial_{j} v_{i} \right]
\end{equation*}
and

\begin{equation*}
  \vb{v}(\div{\vb{v}}) = \vb{e}^{i} v_{i} \partial_{j} v^{j} \,,
\end{equation*}
where \(\varepsilon_{ijk}\) is the Levi-Civita tensor, \(\delta_{i}^{j}\) is the Kroneker-delta symbol and \(\vb{e}^{i}\), \(i=\overline{1,3}\) are the Carthesian versors. Subtracting these two expressions leads to

\begin{equation*}
  \vb{v}(\div{\vb{v}}) - \vb{v}\cp(\curl{\vb{v}}) = \vb{e}^{i} \left[ v_{i} \partial_{j} v^{j} - v^{j} \partial_{i} v_{j} + v^{j} \partial_{j} v_{i} \right] = \vb{e}^{i} \left[ \partial_{j} (v_{i} v^{j}) - v^{j} \partial_{i} v_{j} \right] \,.
\end{equation*}
Since

\begin{equation*}
  \partial_{i} (v^{j} v_{j}) = 2 v^{j} \partial v_{j} \,,
\end{equation*}
we have

\begin{equation*}
  \vb{v}(\div{\vb{v}}) - \vb{v}\cp(\curl{\vb{v}}) = \vb{e}^{i} \left[ \partial_{j} (v_{i} v^{j}) - \frac{1}{2} \partial_{i} (v_{j}v^{j}) \right] \,,
\end{equation*}
The second term can be stylized by introducing a Kroneker-delta and writing \(v_{j}v^{j}\) as \(\vb{v}^{2}\), which leads to the desired final result

\begin{equation}
  \label{expression-2}
  \vb{v}(\div{\vb{v}}) - \vb{v}\cp(\curl{\vb{v}}) = \vb{e}^{i} \partial_{j} \left[ v_{i} v^{j} - \frac{1}{2} \vb{v}^{2} \delta_{i}^{j} \right] = \vb{e}_{i} \partial_{j} \left[ v^{i} v^{j} - \frac{1}{2} \vb{v}^{2} \delta^{ij} \right]\,.
\end{equation}

\par
The last step is possible due to the fact that we only work with space-like components and we chose the convenient metric~(\ref{metric}).
\par
By defining the Maxwell stress tensor as

\begin{equation}
  T^{ij} = \varepsilon_0 \left[ E^{i}E^{j} + c^{2} B^{i}B^{j} - \frac{1}{2} (\vb{E}^{2} + c^{2}\vb{B}^2) \delta^{ij} \right]
\end{equation}
and using it along with~\cref{expression-2} in~\cref{expression-1} one gets

\begin{equation*}
  \vb{e}_{i} \partial_{j} T^{ij} = \varepsilon_0 \vb{E}(\div{\vb{E}}) + \frac{1}{\mu_0} \vb{B}(\div{\vb{B}}) + \vb{j}\cp\vb{B} + \varepsilon_0 \pdv{t}(\vb{E}\cp\vb{B})
\end{equation*}
Using Maxwell's~\cref{eq:coulomb-law,eq:gauss-law} together with the definition of the Poynting vector~(\ref{def:poynting-vector}) the above expression is simplified to the law of momentum conservation

\begin{equation}
  \vb{e}_{i} \partial_{j} T^{ij} = \rho \vb{E} + \vb{j}\cp\vb{B} + \frac{1}{c^{2}} \pdv{\vb{S}}{t}
\end{equation}
or

\begin{equation}
  \vb{e}_{i} \partial_{j} T^{ij} = \rho \vb{E} + \vb{j}\cp\vb{B} + \pdv{\vb{g}}{t}  \,,
\end{equation}
where the volumic density of the fields' electromagnetic momentum \(\vb{g}\) is defined to be

\begin{equation}
  \vb{g} = \frac{1}{c^{2}} \vb{S} = \vb{D}\cp \vb{B} \,.
\end{equation}

\par
By observing that when integrating over a volume, the \(\rho \vb{E} + \vb{j}\cp\vb{B}\) is simply the Lorentz force and that \(\vb{e}_{i} \partial_{j} T^{ij} = \div{\hat{T}}\) we reach an integral form of the momentum conservation

\begin{equation}
  \dv{t} (\vb{P_{em}}+\vb{P_{mech}}) = \int_\mathcal{V} \dd{\vb{r}} \div{\hat{T}} = \int_{\mathcal{S}_\mathcal{V}} \dd{\vb{a}} \vdot \hat{T} \,,
\end{equation}
where \(\vb{P_{em}}\) and \(\vb{P_{mech}}\) are the electromagnetic and mechanical momenta, respectively. If we integrate over the entire space and use the fact that the stress tensor vanishes at infinity, we obtain

\begin{equation}
  \dv{t} (\vb{P_{em}}+\vb{P_{mech}}) = 0 \,.
\end{equation}

\section{Electromagnetic Waves}
The short review of classical electrodynamics had as an ultimate goal to introduce the definitions, equations and formalism required in order to study electromagnetic waves. In this section I will start from the definition and properties of an electromagnetic wave and I will follow up with how one can describe laser pulses. The second part will contain a short introduction to the laser profiles used in research.

\subsection{Maxwell's Equations in Vacuum}
The concept of electromagnetic waves arises naturally from the Maxwell equations~\cref{eq:maxwell-general} if we consider them in vacuum

\begin{subequations}
  \label{eq:maxwell-vacuum}
  \begin{align}
    \div{\vb{D}} & = 0
    \label{eq:coulomb-law-vacuum} \\
    \div{\vb{B}} & = 0
    \label{eq:gauss-law-vacuum} \\
    \curl{\vb{E}} & = - \pdv{\vb{B}}{t}
    \label{eq:faraday-law-vacuum} \\
    \curl{\vb{H}} & = \pdv{\vb{D}}{t}
    \label{eq:ampere-law-vacuum} \,.
  \end{align}
\end{subequations}

\section{Electron Dynamics in Electromagnetic Fields}
\section{The Ponderomotive Force}
\section{Simulations for the Visualization of the Ponderomotive Force}
\end{document}
