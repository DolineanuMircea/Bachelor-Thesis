\documentclass[12pt, class=report, crop=false]{standalone}
\usepackage{ba_thesis}

\begin{document}

% \addcontentsline{toc}{chapter}{Introduction}
\chapter{Electromagnetism and Laser Profiles}%
\label{chap:electromagnetism}

\section{Classical Electrodynamics}

The main principles and laws that govern the phenomena behind lasers, plasma and their interaction are those of classical electrodynamics. As such, like many others tackling this area of research, I find that adding an overview of electrodynamics is simply mandatory. My aim when it comes to differentiating this introductory review from the millions of others out there, if at all possible, is to offer through calculations on some aspects where I personally felt like I wanted to see things from a clearer perspective.

\subsection{Maxwell's Equations}

The Maxwell equations are~(\cite{jacksonClassicalElectrodynamics1999}):

\begin{subequations}
  \begin{align}
    \div{\vb{D}} & = \rho \\
    \curl{\vb{H}} & = \vb{j} + \pdv{\vb{D}}{t} \\
    \curl{\vb{E}} & = - \pdv{\vb{B}}{t} \\
    \div{\vb{B}} & = 0
  \end{align}
\end{subequations}

In the absence of magnetic and polarizable media, \(\vb{D}=\varepsilon_0 \vb{E}\) and \(\vb{B}=\mu_0 \vb{H}\) and the equations become:

\begin{subequations}%
  \label{eq:maxwell}
  \begin{align}
    \div{\vb{E}} & = \frac{\rho}{\varepsilon_0} \label{eq:coulomb-law} \\
    \div{\vb{B}} & = 0 \label{eq:gauss-law} \\
    \curl{\vb{E}} & = - \pdv{\vb{B}}{t} \label{eq:faraday-law} \\
    \curl{\vb{B}} & = \mu_0 \vb{j} + \frac{1}{c^2} \pdv{\vb{E}}{t} \,, \label{eq:ampere-law}
  \end{align}
\end{subequations}

\par
While most readers probably have already had at least a basic introduction to the phenomena from which these equations arise and are well acquainted to how to make use of these equations, I would direct those who haven't towards the book by~\cite{fleischStudentGuideMaxwell2008}%

\par
By extracting the current density from~\cref{eq:ampere-law}, computing its divergence and then replacing the electric field term using~\cref{eq:coulomb-law} one obtains the continuity equation, which relates only the field sources to one another:

\begin{equation}
  \label{eq:continuity-equation}
  \div{\vb{j}(\vb{r}, t)} + \pdv{\rho(\vb{r}, t)}{t} = 0 \,.
\end{equation}

\par
These equations are also complemented by the Lorentz force, which describes how the fields act on the sources. The expression of the Lorentz force in the contiuous case is:

\[
  \vb{F} = \int \dd{\vb{r}'} \left[ \rho(\vb{r}', t) \vb{E}(\vb{r}', t) +
           \frac{1}{c} \vb{j}(\vb{r}', t) \cross \vb{B}(\vb{r}', t)\right] .
\]

\subsection{The Scalar and Vector Potentials}

Since the electric~(\(\vb{E}\)) and magnetic~(\(\vb{B}\)) fields are vectors, they can be described together by a total of six quantities. The sources on the other hand can be described using only four quantities: the electric charge density~\(\rho\) and the three components of the electric current density~\(\vb{j}\). This points to the fact that there is a more convenient way to describe the fields. In finding this alternative, we shall employ the following basic results from algebra:
\begin{subequations}
  \label{id:algebra}
  \begin{align}
    \div{(\curl{\vb{v}})}=0
    \label{id:div-rot} \\
    \curl{(\div{\vb{v}})}=0
    \label{id:rot-div} \\
    \curl{(\grad{f})}=0
    \label{id:rot-grad} \,,
  \end{align}
\end{subequations}
which are valid for any vector function~\(\vb{v}\) and for any scalar function~\(f\).

\par
From~\cref{eq:gauss-law,id:div-rot} one can define the vector potential \(\vb{A}\) such that

\begin{equation}
  \label{eq:vector-potential}
  \vb{B}(\vb{r}, t) = \curl{\vb{A}(\vb{r}, t)}\,.
\end{equation}
By substituting~\eqref{eq:vector-potential} in~\eqref{eq:faraday-law} one obtains

\begin{equation}
  \label{eq:faraday-vector-potential}
  \curl(\vb{E} + \pdv{\vb{A}}{t}) = 0
\end{equation}
which together with~\cref{id:rot-grad} defines the scalar potential \(\phi\)

\begin{equation}
  \label{eq:faraday-scalar-potential}
  \grad{\phi}(\vb{r}, t) = -\vb{E}(\vb{r}, t) - \pdv{\vb{A}}{t}\,.
\end{equation}
Using this in~\cref{eq:coulomb-law}

\begin{equation}
  \label{eq:coulomb-scalar-potential}
  \laplacian{\phi} + \pdv{t}\div{\vb{A}} = - \frac{\rho}{\varepsilon_0}\,.
\end{equation}
Similarly, using~\cref{eq:faraday-scalar-potential} in~\cref{eq:ampere-law} and making use of the following vector identity

\begin{equation}
  \label{eq:curl-of-curl}
  \curl(\curl{\vb{v}}) = \grad(\div{\vb{v}}) - \laplacian{\vb{v}}\,,
\end{equation}
another equation of the potentials is obtained

\begin{equation}
  \label{eq:ampere-potentials}
  \laplacian{\vb{A}} - \frac{1}{c^2}\pdv[2]{\vb{A}}{t} =
    -\mu_0 \vb{j} + \grad(\div{\vb{A}} + \frac{1}{c^2} \pdv{\phi}{t})\,.
\end{equation}

Considering that at every step in the derivation of~\cref{eq:coulomb-scalar-potential,eq:ampere-potentials} we only imposed the Maxwell equations and basic algebraic identities, it follows that~\cref{eq:coulomb-scalar-potential,eq:ampere-potentials} and~\cref{eq:maxwell} are completely equivalent. We now have reduced the six quantities describing the fields to only four: the scalar potential~\(\phi\) and the three components of the vector potential~\(\vb{A}\). This description of the fields through the potentials is quite useful since it is easily integrated in the formalism of special relativity. One can define the electomagnetic potential 4-vector such that the scalar field is the time-like component and the vector field is the space-like component.
\par
In general, when studying the dynamics of particles in an electomagnetic field, once the potentials are computed using~\cref{eq:coulomb-scalar-potential,eq:ampere-potentials} the fields are obtained from~\cref{eq:vector-potential,eq:faraday-scalar-potential} and can be used further in the expression of the Lorentz force.

\subsection{Gauge Transformation}
By a direct application of~\cref{id:algebra} one can show that a simultaneous transformation by an arbitrary well-behaved (continuous with continuous derivatives) scalar function~\(f=f(\vb{r},t)\) of the potentials:

\begin{subequations}
  \label{gauge}
  \begin{align}
    \phi \rightarrow \phi + \pdv{f}{t} \\
    \vb{A} \rightarrow \vb{A} - \grad{f} \,,
  \end{align}
\end{subequations}
leaves the electric and magnetic field unchanged. This is actually a quite natural equivalent of the intuitive fact that any potential is defined up to a constant. In the particular case of the electromagnetic potential,~\cref{gauge} define a gauge transformation. There are two widely used gauges.

\paragraph{Lorenz gauge}

\begin{equation}
  \label{eq:lorenz-gauge}
  \div{\vb{A}} + \frac{1}{c^2} \pdv{\phi}{t}=0
\end{equation}

\par
This gauge cancels the gradient in~\cref{eq:ampere-potentials}. If one works in the usual Mikovski metric~(\cite{weinbergGravitationCosmologyPrinciples1972})

\[
\eta_{\mu \nu} =
\begin{pmatrix}
  -c^{2} & 0 & 0 & 0 \\
  0 & 1 & 0 & 0 \\
  0 & 0 & 1 & 0 \\
  0 & 0 & 0 & 1
\end{pmatrix}
\]
the d'Alembert operator is then defined as
\[
\Box = \partial^{\mu} \partial_{\mu} = \eta^{\mu \nu} \partial_{\nu} \partial_{\mu} = \frac{1}{c^{2}} \pdv[2]{}{t} - \laplacian{}
\]
where \(\mu,\,\nu=\overline{0,3}\) with 0 being the temporal index and 1, 2, 3 being the spatial indices. By replacing this definition in~\cref{eq:coulomb-scalar-potential,eq:ampere-potentials}, it is easy to see that both \(\vb{A}\) and \(\phi\) obey a free wave equation:

\begin{subequations}
  \begin{align}
    \Box \vb{A} = -\mu_0 \vb{j}\\
    \Box \phi = -\frac{\rho}{\varepsilon_0}  \,.
  \end{align}
\end{subequations}

\paragraph{Coulomb Gauge (sometimes found as transversal/velocity gauge)}

\begin{equation}
  \div{\vb{A}} = 0
\end{equation}

\par
Under this gauge, the potential equations~\cref{eq:coulomb-scalar-potential,eq:ampere-potentials} take the form:

\begin{subequations}
  \begin{align}
    \Box \vb{A} = -\mu_0 \vb{j} + \frac{1}{c^{2}} \grad{\pdv{\phi}{t}}\\
    \laplacian{\phi} = -\frac{\rho}{\varepsilon_0}  \,.
  \end{align}
\end{subequations}

\end{document}
