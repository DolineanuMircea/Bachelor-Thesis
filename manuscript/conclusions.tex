\documentclass[12pt, class=report, crop=false]{standalone}
\usepackage{ba_thesis}

\begin{document}

\chapter{Conclusions}%
\label{chap:conclusions}

To sum everything up, this thesis tackled a part of the field of high-intensity laser-plasma interaction, which is one of great interest nowadays given the new high power laser infrastructures that have been constructed in the past couple of years under the ELI project. We have introduced the required theoretical considerations together with examples and throughout calculations. After explaining the formalism used in describing laser beams, the properties that make plasma the perfect candidate for the applications we want to study, and the inner working of the numerical method used, we presented the original results obtained regarding two applications of great importance from both theoretical and experimantal physics: laser wakefield acceleration and the use of structured targets to obtain \(\gamma\)-rays. We have found that at low intensities, in a laser wakefield accelerator, we can obtain multiple beams distributed in a controllable pattern by using Laguerre-Gauss modes. The number of beams is found to be equal to twice the order of the mode (the m number), the cost being a reduction in the maximum energy of the accelerated electron beam of about one order of magnitude. We have also made use of the quantum electrodynamics effects that are available in the PIC code (EPOCH) in order to study the generation of high-energy \(\gamma\)-beams by passage of a high-intensity laser through a structured plasma target shaped like a pipe. We have found that if we keep the exterior radius of the pipe fixed and vary the inside radius, the photon yield increases with the increase in internal radius up to a certain threshhold. This suggest that this is a valid direction in the search for optimizing such technology. The data suggests that the optimal radius should be close to the waist radius of the laser beam.

\end{document}
