\documentclass[12pt, class=report, crop=false]{standalone}
\usepackage{ba_thesis}

\begin{document}

% \addcontentsline{toc}{chapter}{Introduction}
\chapter*{Introduction}%
\addcontentsline{toc}{chapter}{Introduction}
\label{chap:intro}

Considering the new inovations in laser technology and the infrastructures built in the past couple of years to make these available for experimental use, it is now an important time to study and understand the applications that will use those and the physical processes that we will be able to study. The motivation for this work is the developement of the Extreme Light Infrastructure for Nuclear Physics (ELI-NP) facility, which is at the moment is being finalized within the ``Horia~Hulubei'' National Institute for Physics and Nuclear Engineering together with the other two ELI facilities in the Czech Republic and Hungary. In particular, we are interested in the projects related to laser wakefield acceleration (LWFA) using laser-plasma interaction to accelerate electron beams. Such experiments are greatly anticipated, since at the intensities achievable with the new infrastructure QED effects are relevant~(\cite{tanakaCurrentStatusHighlights2020}), and there will be a chance to test various QED theories as well as a possibility to discover new physics. There is also a pormising alternative that uses overdense structured plasma targets to generate photon beams through synchrotron emission. For these purposes, a good understanding of laser-matter interaction together with specific aspects of LWFA is necessary in order to design good experiments and efficiently analyze the results. A powerfull tool in this regard are numerical simulations, or particular to this field of research, Particle-in-Cell codes. These are readily available large scale software that have been used for decades to study plasma related phenomena. We chose to use EPOCH~(\cite{arberContemporaryParticleincellApproach2015}), which is one of the codes that implements multiple ionization mechanisms and has also QED routines available for use.

In this thesis we study the fundamentals of LWFA and offer results from simulations that aim do do a comparative study of using different laser profiles, the main focus being the Laguerre-Gauss modes. We will also present the use of stuctured targets together with normal Gaussian lasers through a couple simulations concerning the study of the photon yield as a function of target geometry. In the first chapter we shortly review basic concepts of electromagnetism and electromagnetic wave theory in order to end up studying mathematical approaches to moddeling laser beams. In the second chapter we discuss light-matter interaction with a great focus on the ponderomotive force, an esential piece in the explanation of LWFA. The third chapter is solely dedicated to understanding plasma specific mechanisms, and provides a few examples. In the fourth chapter we discuss the basic notions regarding numerical simulations of physical systems and then we particularize the ideas in the concrete case of the Particle-in-Cell method. The fifth chapter covers our numerical simulations and results. Finally, we provide a short conclusional chapter that aims to summarize the results obtained. An appendix is added containing various codes used for small numerical studies made along the way in the more theoretical chapters as well as the EPOCH input files used for the simulations presented in the results chapter.

\section*{Acknowledgments}
I would like to thank my adviser, Conf.~Dr.~Alexandru Nicolin, for his help and support, and for providing me with the tools, the sources and the proper environment that I needed in my work. The simulations were performed using the computing cluster at Department of Computational Physics and Information Technologies of the National Institute of Physics and Nuclear Engineering in close collaboration with Dr. Mihaela Carina Raportaru. I would also like to thank the PhD student Sebastian Micluță-Câmpeanu, whithout the help and assistance of whom I would have been lost in the plethora of technical difficulties I have encountered along the way. I also must thank Lect. Univ. Dr. Mădălina Boca  and Conf. Univ. Dr. Mihai Dondera for the fuitful discussions I had with them, which helped me better understand the mathematics and physics behind laser wave modeling.

\end{document}
