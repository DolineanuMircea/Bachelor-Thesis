\documentclass[12pt, class=report, crop=false]{standalone}
\usepackage{ba_thesis}

\begin{document}

\chapter{The Interaction Between Electromagnetic Radiation and Matter}%
\label{chap:light-matter-interaction}
Now that the aspects related to the formalism and theory behind the modeling of laser produced electromagnetic waves has been presented, we must naturally turn our attention towards the interaction of those wave pulses with matter. This chapter only deals with the dinamics of particles under the action of electromagnetic fields and the ponderomotive force, since these topics offer great insight and intuition for the physical behaviour of high intensity laser-plasma interaction. The specific phenomena arising from the properties of plasma as a medium are to be presented later.
\section{Electron Dynamics in Electromagnetic Fields}
This section deals with analyzing the motion of a single electron in the fields of a wave. For simplicity, I will only talk about the case of linearly polarized plane waves, since this entire discussion has the purpose of building up intuition and getting a feel for the scale of the relevant quantities. Most of what is to be presented is following the lecture notes of~\cite{karschApplicationsHighIntensity2018}.

The fact that we want to study dynamics and we are using a very simple type of wave means that it is actually more convenient this time around to work with real fields, rather than complex ones. As per usual, the direction of propagation is chosen to be the z-axis such that the fields are

\begin{subequations}
  \begin{align}
    \vb{E} =\vb{e_x} E_0 \cos(kz-\omega t) \\
    \vb{B} =\vb{e_y} B_0 \cos(kz-\omega t)\,.
  \end{align}
\end{subequations}
Just as an exercise, it can be observed that these fields are generated by the following choice for the 4-potential

\begin{equation}
  \begin{cases}
    \phi = 0 \\
    \vb{A} = \vb{e_x} A_0 \sin(kz-\omega t)\,,
  \end{cases}
\end{equation}
where \(A_0 = \frac{E_0}{\omega} = \frac{B_0}{k}\).
\subsection{Classical Treatment}

We start from the classical equation of motion given by Newton's second principle using the Lorentz force

\begin{equation}
  \dv{\vb{p}}{t} = \dv{(m_e \vb{v_e})}{t} = - e \left( \vb{E} + \vb{v_e}\cp\vb{B} \right)\,,
\end{equation}
with \(m_e\) and \(\vb{v_e}\) the mass and the velocity of the electron, respectively, and \(e\) the elementary charge. Since we have \(B\propto\frac{E}{c}\) and also \(v_e \ll c\) (which is implied in order to have a classical treatment), we can safely remove the second term in the right-hand side of the equation above, remaining with

\begin{equation}
  \dv{\vb{v_e}}{t} = - \frac{e}{m_e} \vb{E} = - \frac{e}{m_e}  E_0 \vb{e_x} \cos(kz-\omega t)\,.
\end{equation}
Simply integrating with initial conditions \(x_0\), \(y_0\), \(z_0\) \(=0\) and \(\vb{v_e(0)} = \vb{0}\) leads to

\begin{subequations}
  \begin{align}
    \vb{v_e} (t) = \frac{e}{\omega m_e} E_0 \vb{e_x} \sin(kz-\omega t) \\
    x (t) = \frac{e}{\omega^2 m_e} E_0 \left[ \cos(kz-\omega t) -1 \right]\,.
  \end{align}
\end{subequations}


\section{The Ponderomotive Force}
\section{Simulations for the Visualization of the Ponderomotive Force}
\section{Laser Wakefield Acceleration}
\end{document}
