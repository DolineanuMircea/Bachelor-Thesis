\documentclass[12pt, class=report, crop=false]{standalone}
\usepackage{ba_thesis}

\begin{document}

\chapter{The Interaction Between Electromagnetic Radiation and Matter}%
\label{chap:light-matter-interaction}
Now that the aspects related to the formalism and theory behind the modeling of laser produced electromagnetic waves has been presented, we must naturally turn our attention towards the interaction of those wave pulses with matter. This chapter only deals with the dinamics of particles under the action of electromagnetic fields and the ponderomotive force, since these topics offer great insight and intuition for the physical behaviour of high intensity laser-plasma interaction. The specific phenomena arising from the properties of plasma as a medium are to be presented later.
\section{Electron Dynamics in Electromagnetic Fields}
This section deals with analyzing the motion of a single electron in the fields of a wave. For simplicity, I will only talk about the case of linearly polarized plane waves, since this entire discussion has the purpose of building up intuition and getting a feel for the scale of the relevant quantities. Most of what is to be presented is following the lecture notes of~\cite{karschApplicationsHighIntensity2018}.

The fact that we want to study dynamics and we are using a very simple type of wave means that it is actually more convenient this time around to work with real fields, rather than complex ones. As per usual, the direction of propagation is chosen to be the z-axis such that the fields are

\begin{subequations}
  \begin{align}
    \vb{E} =\vb{e_x} E_0 \cos(kz-\omega t) \\
    \vb{B} =\vb{e_y} B_0 \cos(kz-\omega t)\,.
  \end{align}
\end{subequations}
Just as an exercise, it can be observed that these fields are generated by the following choice for the 4-potential

\begin{equation}
  \label{choice:potentials}
  \begin{cases}
    \phi = 0 \\
    \vb{A} = \vb{e_x} A_0 \sin(kz-\omega t)\,,
  \end{cases}
\end{equation}
where \(A_0 = \frac{E_0}{\omega} = \frac{B_0}{k}\).
\subsection{Classical Treatment}

We start from the classical equation of motion given by Newton's second principle using the Lorentz force

\begin{equation}
  \label{eq:equation-of-motion-electron}
  \dv{\vb{p}}{t} = \dv{(m_e \vb{v_e})}{t} = - e \left( \vb{E} + \vb{v_e}\cp\vb{B} \right)\,,
\end{equation}
with \(m_e\) and \(\vb{v_e}\) the mass and the velocity of the electron, respectively, and \(e\) the elementary charge. Since we have \(B\propto\frac{E}{c}\) and also \(v_e \ll c\) (which is implied in order to have a classical treatment), we can safely remove the second term in the right-hand side of the equation above, remaining with

\begin{equation}
  \dv{\vb{v_e}}{t} = - \frac{e}{m_e} \vb{E} = - \frac{e}{m_e}  E_0 \vb{e_x} \cos(kz-\omega t)\,.
\end{equation}
Simply integrating with initial conditions \(x_0\), \(y_0\), \(z_0\) \(=0\) and \(\vb{v_e(0)} = \vb{0}\) leads to

\begin{subequations}
  \begin{align}
    \vb{v_e} (t) = \frac{e}{\omega m_e} E_0 \vb{e_x} \sin(kz-\omega t) \\
    x (t) = \frac{e}{\omega^2 m_e} E_0 \left[ \cos(kz-\omega t) -1 \right]\,.
  \end{align}
\end{subequations}

It is important now to see when the classical treatment breaks down. Let us impose that

\begin{equation}
  v_e^{max} = c \,,
\end{equation}
such that we have

\begin{equation}
  a_0 \equiv \frac{e E_0}{\omega m_e c} = \frac{e A_0}{m_e c} = 1\,.
\end{equation}
The parameter \(a_0\) is called the normalized or dimensionless vector potential. From its definition it is easy to see that it can only take values between 0 and 1. We can use it to describe the amplitude of the electric field as such

\begin{equation}
  E_0 = a_0 \frac{\omega m_e c}{e}\,.
\end{equation}
It is very convenient in practice to use the wavelength and to extract the rest mass to charge ratio of the electron as follows

\begin{equation}
  E_0 = \frac{a_0}{\lambda} 2\pi \frac{m_e c^2}{e} = \frac{a_0}{\lambda} 2\pi\cdot 511 kV\,.
\end{equation}
The normalized vector field helds also an important significance. One can see that its definition actually boils down to

\begin{equation}
  a_0 = \frac{v_{max}^{classical}}{c}\,,
\end{equation}
so we can use it to find a boundary for the validity of the classical treatment. For simplicity, lets disect the \(a_0 = 1\) case, for which the motion should be completely relativistic, keeping in mind that the classical description stops being reliable well befor that. From the result concerning the Poynting vector of a plane wave~(\ref{def:poynting-plane-wave}), we can find the intensity of the pulse in this limiting case to be

\begin{equation}
  I = c \frac{\varepsilon_0}{2} E_0^2 \propto \frac{a_0^2}{\lambda^2} 10^{18} W \frac{\mu m^2}{cm^2}\;,
\end{equation}
which says that already at intensities of \(10^{18} \frac{W}{cm^2}\) the motion of the electron should be treated completely within the grounds of special relativity.

\subsection{Relativistic Treatment}
In the light of our discussion in the previous subsection, we see that in order to study how electrons interact with high-intensity laser beams (namely, terwatt and petawatt lasers), we should do all our calculations relativistically. The equation of motion remains the same, but the relativistic momentum is \(\vb{p} = \gamma m_e \vb{v_e}\), where \(\gamma\) is the usual Lorentz factor. By taking the scalar product of~\cref{eq:equation-of-motion-electron} with \(\vb{p}\) we get

\begin{equation}
  \frac{1}{2} \dv{\vb{p}^2}{t} = - e \vb{p}\vdot\vb{E}\,,
\end{equation}
where we used the fact that \(\vb{p}\vdot\left(\vb{v_e}\cp\vb{B}\right)\), since \(\vb{p}\) is proportional to \(\vb{v_e}\). Now, it is useful to write the Lorenz factor in terms of momentum like this

\begin{equation*}
  \gamma = \frac{1}{\sqrt{1 - \frac{\vb{v}^2}{c^2}}} \Rightarrow
  \frac{1}{\gamma^2} = 1 - \frac{\vb{v}^2}{c^2} = 1 - \frac{1}{\gamma^2} \frac{\vb{p}^2}{m_e^2 c^2} \Rightarrow \gamma = \sqrt{1+\left(\frac{\vb{p}}{m_e c}\right)^2} \,.
\end{equation*}
Now we can expect to find a \(\dv{\vb{p}^2}{t}\) by taking the derivative of \(\gamma\) with respect to time

\begin{equation*}
  \dv{\gamma}{t} = \dv{t} \sqrt{1+\left(\frac{\vb{p}}{m_e c}\right)^2} = \frac{1}{\sqrt{1+\left(\frac{\vb{p}}{m_e c}\right)^2} } \frac{1}{m_e^2 c^2} \frac{1}{2} \dv{\vb{p}^2}{t} = \frac{1}{\gamma m_e^2 c^2}\frac{1}{2} \dv{\vb{p}^2}{t} \Rightarrow
\end{equation*}

\begin{equation*}
  \Rightarrow m_e c^2 \dv{\gamma}{t} = - e \frac{\vb{p}}{\gamma m_e} \vdot \vb{E} = -e \vb{v_e}\vdot\vb{E}\,.
\end{equation*}
Remembering that the kinetic energy in special relativity is obtained as \(K = (\gamma - 1) m_e c^2\), we can reach an equation for it

\begin{equation}
  \dv{K}{t} = -e \vb{v_e}\vdot\vb{E}\,.
\end{equation}
This equation can also be rewritten as

\begin{equation}
  \label{eq:gamma-vx}
  \dv{\gamma}{t} = -\frac{e E_0}{m_e c} \frac{\vb{v_e}\vdot\vb{e_x}}{c} \cos(kz-\omega t) = - a_0 \omega \frac{v_x}{c} \cos(kz-\omega t)\,.
\end{equation}
In order to procees, we should also take a better look at the equation of motion as it is

\begin{equation*}
  \dv{t} (\frac{\vb{p}}{m_e c}) = - \frac{e}{m_e c} \left[E_0 \vb{e_x} + B_0 \vb{v}\cp\vb{e_y}\right]\cos(kz-\omega t) =
\end{equation*}

\begin{equation*}
  = - \frac{e E_0}{m_e c} \left[\vb{e_x} + \frac{\vb{v}}{c}\cp\vb{e_y} \right]\cos(kz-\omega t) =
\end{equation*}

\begin{equation*}
  = -a_0 \omega \left[ \left(1 - \frac{v_z}{c} \right)\vb{e_x} + \frac{v_x}{c} \vb{e_z} \right]\cos(kz-\omega t)\,.
\end{equation*}
By defining \(\vb{\tilde{p}} = \frac{\vb{p}}{m_e c}\) to be the normalized momentum, we can write the equations for its components as follows

\begin{subequations}
  \begin{flalign}
    \label{eq:momentum-x}
    \dv{\tilde{p_x}}{t} = -a_0 \omega \left(1 - \frac{v_z}{c} \right) \cos(kz-\omega t) \\
    \dv{\tilde{p_y}}{t} = 0 \\
    \label{eq:momentum-z}
    \dv{\tilde{p_z}}{t} = -a_0 \omega \frac{v_x}{c} \cos(kz-\omega t)\,.
  \end{flalign}
\end{subequations}
The \(y\) is trivial, and since we took the initial velocity to be zero, the \(y\)-velocity will be zero at all times.

From~\cref{eq:gamma-vx,eq:momentum-z} we have the following developement

\begin{equation}
  \dv{(\tilde{p_z}-\gamma)}{t} = 0 \Leftrightarrow \tilde{p_z}-\gamma = C \,.
\end{equation}
Again, making use of our choice of initial conditions, which translate here as \(\gamma (0) = 1\) and \(\tilde{p_z} (0)= 0\), we obtain \(C\) to be -1.

To summarize so far, we already know that

\begin{subequations}
  \begin{align}
    \tilde{p_y} = 0\\
    \label{sol:pz2}
    \tilde{p_z} = \gamma - 1\,.
  \end{align}
\end{subequations}
By squaring up the very last equation, yet another useful relation can be derived

\begin{equation*}
  \gamma = 1 + \tilde{p_z} = \sqrt{1+\vb{\tilde{p}}} \Rightarrow 1+2\tilde{p_z}+ \tilde{p_z}^2 = 1+\tilde{p_x}+\tilde{p_y}+\tilde{p_z}
\end{equation*}

\begin{equation}
  \label{sol:pz}
  \tilde{p_z} =  \frac{1}{2} \tilde{p_x}^2\,.
\end{equation}
Since \(p_z\) is the normalized momentum along the propagation direction of the wave, we can see that for \(\tilde{p_z} =  \frac{1}{2} \tilde{p_x}^2 \ll 1\) (classical regime) the transversal momentum is more important, while for \(\tilde{p_z} =  \frac{1}{2} \tilde{p_x}^2 \gg 1\) (highly relativistic regime) the longitudinal momentum is more important.

Now we would like to find \(p_x\). There is actually a more efficient way to do so than solving~\cref{eq:momentum-x,eq:momentum-z} together. But for that we shall make use of the electromagnetic potential~(\ref{choice:potentials}). We also need the relations that define the fields from the potential, as detailed in section~(\ref{ch-ref-1}). With this in mind, we return to the equation of motion

\begin{equation}
  \dv{\vb{p}}{t} = - e \left( \vb{E} + \vb{v_e}\cp\vb{B} \right) = -e \left( -\partial_t \vb{A} + \vb{v_e}\cp(\curl{\vb{A}}) \right)\,.
\end{equation}
We make use of the vector identity

\begin{equation}
  \vb{v}\cp(\curl{\vb{u}}) = \grad{(\vb{v}\vdot\vb{u})} - (\vb{v}\vdot \grad) \vb{u}\,,
\end{equation}
and the total derivative of \(A\) with respect to time

\begin{equation}
  \dv{\vb{A}}{t} =
  \partial_t \vb{A} + \partial_{x_i} A^{x_i} \partial_t x_i =
  \pdv{\vb{A}}{t} + (\vb{v_e}\vdot \grad) \vb{A}\,.
\end{equation}
Thus,

\begin{equation}
  \dv{\vb{p}}{t} = - e \left[ -\dv{\vb{A}}{t} + (\vb{v_e}\vdot \grad) \vb{A} + \grad{(\vb{v_e}\vdot\vb{A})} - (\vb{v_e}\vdot \grad) \vb{A} \right] = - e \left[ -\dv{\vb{A}}{t} + \grad{(\vb{v_e}\vdot\vb{A})} \right]\,.
\end{equation}
Since

\begin{equation*}
  \dv{\vb{A}}{t} = - \vb{e_x} \omega A_0 \cos(kz-\omega t)
\end{equation*}

\begin{equation*}
  \grad{(\vb{v_e}\vdot\vb{A})} = \vb{e_z} k v_x A_0 \cos(kz-\omega t)\,,
\end{equation*}
we can extract the \(p_x\) equation

\begin{equation}
  \dv{p_x}{t} = - e \omega A_0 \cos(kz-\omega t) = e \dv{A}{t}\,,
\end{equation}
which gives

\begin{equation}
  p_x = eA + C'\,.
\end{equation}
With the initial conditions this becomes

\begin{equation}
  \label{sol:px}
  p_x = e A = e A_0 \sin(kz-\omega t)\,.
\end{equation}

Bringing together~\cref{sol:px,sol:pz,sol:pz2}, using the normalized vector potential \(a = \frac{e A_0}{m_e c} \sin(kz-\omega t)\)

\begin{equation}
  K = (\gamma -1)m_e c^2 =\frac{a^2}{2}m_e c^2 \Rightarrow \gamma = 1 + \frac{a^2}{2}\,.
\end{equation}

With this, all the puzzle pieces are in place, so we can collect the following results concerning the motion of the electron

\begin{subequations}
  \begin{align}
    \gamma = 1 + \frac{a^2}{2}\\
    \tilde{p_x} = a\\
    \tilde{p_y} = 0\\
    \tilde{p_z} = \frac{a^2}{2}\,.
  \end{align}
\end{subequations}
It is more convenient, and general, to work with the derivative with respect to \(\tau = t - \frac{z(t)}{c}\), which is to choose the convenient reference frame of the wave to simplify the computations. This derivative is developed as such

\begin{equation}
  \dv{t} = \dv{\tau}{t} \dv{\tau} = \left( 1 - \frac{1}{c} \dv{z}{t} \right) \dv{\tau} = \left( 1 - \frac{a^2}{2}  \right) \dv{\tau} = \frac{1}{\gamma} \dv{\tau}\,.
\end{equation}
We can also write the phase of \(a\) in terms of this time

\begin{equation}
  kz-\omega  t= kz -\omega\tau - \frac{\omega}{c} z = - \omega \tau\,.
\end{equation}
A simple substitution gives the equations for the coordinates

\begin{subequations}
  \begin{align}
    \dv{x}{\tau} = c a\\
    \dv{y}{\tau} = 0\\
    \dv{z}{\tau} = c \frac{a^2}{2}\,,
  \end{align}
\end{subequations}
which have the solutions (of course using the initial conditions we chose at the begining)

\begin{subequations}
  \begin{align}
    x(\tau) = \frac{ca_0}{\omega} \left[\cos(\omega \tau) - 1 \right] \\
    y(\tau) = 0\\
    z(\tau) = \frac{ca_0^2}{4} \left[\tau - \frac{1}{2\omega} \sin(2\omega \tau)\right]\,.
  \end{align}
\end{subequations}

These results show us that the motion in the transversal plane is the same as in the classical motion. However, the motion on the propagation direction is more complex, being a superposition of an oscillation and a drift motion. The drift velocity \(v_{drift} = \expval{v_z} = \expval{\frac{z}{t}}\) can be computed fairly easy

\begin{equation*}
  z_{drift} = \frac{ca_0^2}{4} \tau = \frac{ca_0^2}{4} \left( t - \frac{z}{c} \right) = \frac{ca_0^2}{4} t - \frac{a_0^2}{4} z_{drift}
\end{equation*}

\begin{equation*}
  z_{drift} = \frac{ca_0^2}{4} \frac{t}{1+\frac{a_0^2}{4}} = \frac{ca_0^2}{4+a_0^2} t
\end{equation*}

\begin{equation}
  v_{drift} = \frac{ca_0^2}{4+a_0^2}\,.
\end{equation}


\section{The Ponderomotive Force}
\section{Simulations for the Visualization of the Ponderomotive Force}
\section{Laser Wakefield Acceleration}
\end{document}
