\documentclass[12pt, class=report, crop=false]{standalone}
\usepackage{ba_thesis}

\begin{document}

\chapter{The Interaction Between Electromagnetic Radiation and Matter}%
\label{chap:light-matter-interaction}
Now that the aspects related to the formalism and theory behind the modeling of laser produced electromagnetic waves has been presented, we must naturally turn our attention towards the interaction of those wave pulses with matter. This chapter only deals with the dinamics of particles under the action of electromagnetic fields and the ponderomotive force, since these topics offer great insight and intuition for the physical behaviour of high intensity laser-plasma interaction. The specific phenomena arising from the properties of plasma as a medium are to be presented later.
\section{Electron Dynamics in Electromagnetic Fields}
This section deals with analyzing the motion of a single electron in the fields of a wave. For simplicity, I will only talk about the case of linearly polarized plane waves, since this entire discussion has the purpose of building up intuition and getting a feel for the scale of the relevant quantities. Most of what is to be presented is following the lecture notes of~\cite{karschApplicationsHighIntensity2018}.

The fact that we want to study dynamics and we are using a very simple type of wave means that it is actually more convenient this time around to work with real fields, rather than complex ones. As per usual, the direction of propagation is chosen to be the z-axis such that the fields are

\begin{subequations}
  \begin{align}
    \vb{E} =\vb{e_x} E_0 \cos(kz-\omega t) \\
    \vb{B} =\vb{e_y} B_0 \cos(kz-\omega t)\,.
  \end{align}
\end{subequations}
Just as an exercise, it can be observed that these fields are generated by the following choice for the 4-potential

\begin{equation}
  \begin{cases}
    \phi = 0 \\
    \vb{A} = \vb{e_x} A_0 \sin(kz-\omega t)\,,
  \end{cases}
\end{equation}
where \(A_0 = \frac{E_0}{\omega} = \frac{B_0}{k}\).
\subsection{Classical Treatment}

We start from the classical equation of motion given by Newton's second principle using the Lorentz force

\begin{equation}
  \label{eq:equation-of-motion-electron}
  \dv{\vb{p}}{t} = \dv{(m_e \vb{v_e})}{t} = - e \left( \vb{E} + \vb{v_e}\cp\vb{B} \right)\,,
\end{equation}
with \(m_e\) and \(\vb{v_e}\) the mass and the velocity of the electron, respectively, and \(e\) the elementary charge. Since we have \(B\propto\frac{E}{c}\) and also \(v_e \ll c\) (which is implied in order to have a classical treatment), we can safely remove the second term in the right-hand side of the equation above, remaining with

\begin{equation}
  \dv{\vb{v_e}}{t} = - \frac{e}{m_e} \vb{E} = - \frac{e}{m_e}  E_0 \vb{e_x} \cos(kz-\omega t)\,.
\end{equation}
Simply integrating with initial conditions \(x_0\), \(y_0\), \(z_0\) \(=0\) and \(\vb{v_e(0)} = \vb{0}\) leads to

\begin{subequations}
  \begin{align}
    \vb{v_e} (t) = \frac{e}{\omega m_e} E_0 \vb{e_x} \sin(kz-\omega t) \\
    x (t) = \frac{e}{\omega^2 m_e} E_0 \left[ \cos(kz-\omega t) -1 \right]\,.
  \end{align}
\end{subequations}

It is important now to see when the classical treatment breaks down. Let us impose that

\begin{equation}
  v_e^{max} = c \,,
\end{equation}
such that we have

\begin{equation}
  a_0 \equiv \frac{e E_0}{\omega m_e c} = \frac{e A_0}{m_e c} = 1\,.
\end{equation}
The parameter \(a_0\) is called the normalized or dimensionless vector potential. From its definition it is easy to see that it can only take values between 0 and 1. We can use it to describe the amplitude of the electric field as such

\begin{equation}
  E_0 = a_0 \frac{\omega m_e c}{e}\,.
\end{equation}
It is very convenient in practice to use the wavelength and to extract the rest mass to charge ratio of the electron as follows

\begin{equation}
  E_0 = \frac{a_0}{\lambda} 2\pi \frac{m_e c^2}{e} = \frac{a_0}{\lambda} 2\pi\cdot 511 kV\,.
\end{equation}
The normalized vector field helds also an important significance. One can see that its definition actually boils down to

\begin{equation}
  a_0 = \frac{v_{max}^{classical}}{c}\,,
\end{equation}
so we can use it to find a boundary for the validity of the classical treatment. For simplicity, lets disect the \(a_0 = 1\) case, for which the motion should be completely relativistic, keeping in mind that the classical description stops being reliable well befor that. From the result concerning the Poynting vector of a plane wave~(\ref{def:poynting-plane-wave}), we can find the intensity of the pulse in this limiting case to be

\begin{equation}
  I = c \frac{\varepsilon_0}{2} E_0^2 \propto \frac{a_0^2}{\lambda^2} 10^{18} W \frac{\mu m^2}{cm^2}\;,
\end{equation}
which says that already at intensities of \(10^{18} \frac{W}{cm^2}\) the motion of the electron should be treated completely within the grounds of special relativity.

\subsection{Relativistic Treatment}
In the light of our discussion in the previous subsection, we see that in order to study how electrons interact with high-intensity laser beams (namely, terwatt and petawatt lasers), we should do all our calculations relativistically. The equation of motion remains the same, but the relativistic momentum is \(\vb{p} = \gamma m_e \vb{v_e}\), where \(\gamma\) is the usual Lorentz factor. By taking the scalar product of~\cref{eq:equation-of-motion-electron} with \(\vb{p}\) we get

\begin{equation}
  \frac{1}{2} \dv{\vb{p}^2}{t} = - e \vb{p}\vdot\vb{E}\,,
\end{equation}
where we used the fact that \(\vb{p}\vdot\left(\vb{v_e}\cp\vb{B}\right)\), since \(\vb{p}\) is proportional to \(\vb{v_e}\). Now, it is useful to write the Lorenz factor in terms of momentum like this

\begin{equation}
  do\; to
\end{equation}

\section{The Ponderomotive Force}
\section{Simulations for the Visualization of the Ponderomotive Force}
\section{Laser Wakefield Acceleration}
\end{document}
