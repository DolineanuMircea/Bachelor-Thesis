\documentclass[12pt, class=report, crop=false]{standalone}
\usepackage{ba_thesis}

\begin{document}

\chapter*{Appendix}%
\addcontentsline{toc}{chapter}{Appendix}

\section*{Small Simulation of Electron Dynamics in a Laser Pulse}

\lstinputlisting{../simulations/electron-motion.jl}

\newpage
\section*{Implementation of a Few Imtegrators for the Harmonic Oscillator for the Visualization of Stability}

\lstinputlisting{../simulations/stability.jl}
% \begin{lstlisting}
% # some julia code
% println( "Here we go with Julia!")
% \end{lstlisting}

\section*{EPOCH input.deck for low intensity simulations (m=1, p=0)}

\begin{verbatim}
begin:constant
  lambda0 = 1.0 * micron
  temp = 0.0
  a0 = 2 #normalized vector potential
  dens = critical(2 * pi * c / lambda0) / 1000 #approx 1.0e24, ncrit=1.11*10^27
  r = sqrt(y^2 + z^2)
  phi = atan2(y,z)
  w_0 = 5 * micron
  m = 1
  p = 0
  Cpm = 1 #sqrt(p!/(m+p)!)
end:constant

begin:control
  nx = 1200
  ny = 300
  nz = 300

  # final time of simulation
  t_end = 400 * femto

  # size of domain
  x_min = -30.0 * lambda0
  x_max = -x_min
  y_min = -15.0 * lambda0
  y_max = -y_min
  z_min = y_min
  z_max = -z_min

  stdout_frequency = 60
end:control

begin:boundaries
  bc_x_min = simple_laser
  bc_x_max = simple_outflow
  bc_y_min = periodic
  bc_y_max = periodic
  bc_z_min = periodic
  bc_z_max = periodic
end:boundaries

begin:laser
  boundary = x_min
  lambda = lambda0
  amp = a0 / lambda0 * 2 * pi * me*c^2/qe # E0
  profile = Cpm * (sqrt(2)*r/w_0)^m * gauss(r,0,w_0) * cos(m*phi)
  #t_start = 0.0
  #t_end = 17.0 * femto
  t_profile = gauss(time,34*femto,17*femto)
end:laser

begin:species
  name = electron
  charge = -1.0
  mass = 1.0
  #frac = 1.0
  temp = 0
  density = dens
  npart_per_cell = 5
end:species

begin:species
  #He ions
  name = ions
  charge = 2.0
  mass = 4.0 * 1830
  npart_per_cell = 2
  immobile = T
  density = 0.5 * dens
  temp = 0.0
end:species

begin:window
  move_window = T
  window_v_x = c
  #when pulse is centered in window
  window_start_time = (x_max - x_min)/c
  bc_x_min_after_move = simple_outflow
  bc_x_max_after_move = simple_outflow
end:window

begin:output
  #timesteps between output dumps
  dt_snapshot = 10.0 * femto

  # Properties on grid
  grid = always
  ex = always
  ey = always
  ez = always

  # Properties at particle positions
  particle_grid = always

  number_density = always + species
  charge_density = always + species
  average_particle_energy = always + species
end:output
\end{verbatim}

\section*{EPOCH input.deck for higher intensity simulations (m=1, p=0)}

\begin{verbatim}
begin:control
  nx = 1000
  ny = 250
  nz = 250

  # Final time of simulation
  t_end = 350 * femto

  # Size of domain
  x_min = -50 * 800 * nano
  x_max = -x_min
  y_min = -25 * 800 * nano
  y_max = -y_min
  z_min = y_min
  z_max = -z_min

  stdout_frequency = 60
end:control


begin:boundaries
  bc_x_min = simple_laser
  bc_x_max = open
  bc_y_min = periodic
  bc_y_max = periodic
  bc_z_min = periodic
  bc_z_max = periodic
end:boundaries


begin:constant
  lambda0 = 800 * nano
  omega = 2 * pi * c / lambda0
  T_l = lambda0 / c

  a0 = 70
  w_0 = 7 * lambda0 # Beam waist size
  r = sqrt(y^2 + z^2)
  phi = atan2(y,z)

  m = 1
  p = 0
  Cpm = 1 #sqrt(p!/(m+p)!)
  n = 2.4e26
end:constant


begin:laser
  boundary = x_min
  lambda = lambda0
  amp = a0 / lambda0 * 2 * pi * me*c^2/qe # E0
  phase = m*phi
  profile = Cpm * (sqrt(2)*r/w_0)^m * gauss(r,0,w_0)
  t_profile = gauss(time, 0.0, 20*T_l)
end:laser


begin:species
   name = electron
   charge = -1.0
   mass = 1.0
   number_density = if((-45*lambda0 lt x) and (x lt 50*lambda0), n, 0.0)
   temperature_ev = 1e3
   nparticles_per_cell = 4
end:species


begin:species
   name = helium
   charge = 2.0
   mass = 7296.3
   number_density = 0.5*number_density(electron)
   temperature_ev = 30
   nparticles_per_cell = 2
end:species


begin:output
  name = normal

  # Number of timesteps between output dumps
  dt_snapshot = 15 * femto

  # Properties on grid
  grid = always
  ex = always
  ey = always
  ez = always
  poynting_flux = always

  # Properties at particle positions
  particle_grid = always
  px = always
  py = always
  pz = always

  number_density = always + species
  average_particle_energy = always + species
  particle_energy_flux = always + species
end:output

\end{verbatim}

\section*{Structured target}
There is a short remark to be made. This input.deck uses QED effects. By default, all the QED effects in EPOCH are turned off from inside the code. So in order to use them, you will need to make changes to the makefile before compiling the EPOCH code. The documentation and other documents that mention this don't give you a clear explanation on how to do it without bothering to look at what is inside the makefile (at least at the moment of writing this - epoch-v4.17.10). For any reader who wants to just use the software with absolutely zero interaction with what is written in the makefile, just use the following linux commands
\begin{verbatim}
  user@home:~# cd /$EPOCH_VERSION/epoch1d
  user@home:~# sed -i 's/#DEFINES += $(D)PHOTONS/DEFINES += $(D)PHOTONS/g' Makefile
  user@home:~# make clean
\end{verbatim}
before compiling the code.

\begin{verbatim}
begin:constant
  #laser parameters
  lambda0 = 1.0 * micron
  a0 = 190
  w0 = 1.3 * micron #change it to change power

  #plasma parameters
  n_cr = critical(2 * pi *c / lambda0)
  n_bulk = 100 * n_cr
  n_ch = 20 * n_cr

  d_bulk = 3 * micron
  R_ch = 0.7 * w0
  L_ch = 33 * micron

  #general
  r = sqrt(y^2 + z^2)
  x0 = 0.0
end:constant

begin:control
  #grid
  nx = 990
  ny = 240
  nz = 240

  #simulation time
  t_end = 110 * femto

  #domain size

  x_min = 0.0
  x_max = L_ch
  y_min = - 4.0 * micron
  y_max = -y_min
  z_min = y_min
  z_max = -y_min

  stdout_frequency = 20
end:control

begin:boundaries
  bc_x_min = simple_laser
  bc_x_max = simple_outflow
  bc_y_min = periodic
  bc_y_max = periodic
  bc_z_min = periodic
  bc_z_max = periodic
end:boundaries

begin:laser
  boundary = x_min
  lambda = lambda0
  amp = a0 / lambda0 * 2 * pi * me*c^2/qe

  profile = gauss(r, 0.0, w0)

  t_profile = gauss(time, 0.0, 35*femto)
end:laser

begin:qed
  use_qed = T
  qed_start_time = 0
  produce_photons = T
  qed_table_location = /epoch-4.17.10/epoch3d/src/physics_packages/TABLES
end:qed

begin:species
  #synchroton photons
  name = photon
  nparticles = 0
  dump = T
  identify:photon
end:species

begin:species
  #electrons
  name = electron
  # charge = -1.0
  # mass = 1.0
  temp = 0.0
  dump = T
  npart_per_cell = 15

  number_density = if(r lt (R_ch + d_bulk), n_bulk, 0.0)
  number_density = if(r lt R_ch, n_ch, number_density(electron))
  number_density = if((x lt x0) and (x gt L_ch), 0.0, number_density(electron))

  #for QED effects to work
  identify:electron
end:species

begin:species
  #carbon ions
  name = carbon
  charge = 6.0
  mass = 12 * 1836.2
  temp = 0.0
  dump = T
  npart_per_cell = 5

  number_density = number_density(electron) / 6
end:species

begin:output
  #timesteps between output dumps
  dt_snapshot = 10.0 * femto

  # Properties on grid
  grid = always

  # Properties at particle positions
  particle_grid = always
  #px = always + species
  #py = always + species
  #pz = always + species
  #momenta and particle_energy are not defined for photons

  number_density = always + species
  average_particle_energy = always + species
end:output

\end{verbatim}

\end{document}
