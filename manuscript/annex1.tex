\documentclass[12pt, class=report, crop=false]{standalone}
\usepackage{ba_thesis}

\begin{document}

\chapter*{Appendix}%
\addcontentsline{toc}{chapter}{Appendix}

\section*{Small Simulation of Electron Dynamics in a Laser Pulse}

\lstinputlisting{../simulations/electron-motion.jl}

\newpage
\section*{Implementation of a Few Imtegrators for the Harmonic Oscillator for the Visualization of Stability}

\lstinputlisting{../simulations/stability.jl}
% \begin{lstlisting}
% # some julia code
% println( "Here we go with Julia!")
% \end{lstlisting}

\section*{EPOCH input.deck for low intensity simulations (m=1, p=0)}

\begin{verbatim}
begin:constant
  lambda0 = 1.0 * micron
  temp = 0.0
  a0 = 2 #normalized vector potential
  dens = critical(2 * pi * c / lambda0) / 1000 #approx 1.0e24, ncrit=1.11*10^27
  r = sqrt(y^2 + z^2)
  phi = atan2(y,z)
  w_0 = 5 * micron
  m = 1
  p = 0
  Cpm = 1 #sqrt(p!/(m+p)!)
end:constant

begin:control
  nx = 1200
  ny = 300
  nz = 300

  # final time of simulation
  t_end = 400 * femto

  # size of domain
  x_min = -30.0 * lambda0
  x_max = -x_min
  y_min = -15.0 * lambda0
  y_max = -y_min
  z_min = y_min
  z_max = -z_min

  stdout_frequency = 60
end:control

begin:boundaries
  bc_x_min = simple_laser
  bc_x_max = simple_outflow
  bc_y_min = periodic
  bc_y_max = periodic
  bc_z_min = periodic
  bc_z_max = periodic
end:boundaries

begin:laser
  boundary = x_min
  lambda = lambda0
  amp = a0 / lambda0 * 2 * pi * me*c^2/qe # E0
  profile = Cpm * (sqrt(2)*r/w_0)^m * gauss(r,0,w_0) * cos(m*phi)
  #t_start = 0.0
  #t_end = 17.0 * femto
  t_profile = gauss(time,34*femto,17*femto)
end:laser

begin:species
  name = electron
  charge = -1.0
  mass = 1.0
  #frac = 1.0
  temp = 0
  density = dens
  npart_per_cell = 5
end:species

begin:species
  #He ions
  name = ions
  charge = 2.0
  mass = 4.0 * 1830
  npart_per_cell = 2
  immobile = T
  density = 0.5 * dens
  temp = 0.0
end:species

begin:window
  move_window = T
  window_v_x = c
  #when pulse is centered in window
  window_start_time = (x_max - x_min)/c
  bc_x_min_after_move = simple_outflow
  bc_x_max_after_move = simple_outflow
end:window

begin:output
  #timesteps between output dumps
  dt_snapshot = 10.0 * femto

  # Properties on grid
  grid = always
  ex = always
  ey = always
  ez = always

  # Properties at particle positions
  particle_grid = always

  number_density = always + species
  charge_density = always + species
  average_particle_energy = always + species
end:output
\end{verbatim}

\section*{EPOCH input.deck for higher intensity simulations (m=1, p=0)}

\begin{verbatim}
begin:control
  nx = 1000
  ny = 250
  nz = 250

  # Final time of simulation
  t_end = 350 * femto

  # Size of domain
  x_min = -50 * 800 * nano
  x_max = -x_min
  y_min = -25 * 800 * nano
  y_max = -y_min
  z_min = y_min
  z_max = -z_min

  stdout_frequency = 60
end:control


begin:boundaries
  bc_x_min = simple_laser
  bc_x_max = open
  bc_y_min = periodic
  bc_y_max = periodic
  bc_z_min = periodic
  bc_z_max = periodic
end:boundaries


begin:constant
  lambda0 = 800 * nano
  omega = 2 * pi * c / lambda0
  T_l = lambda0 / c

  a0 = 70
  w_0 = 7 * lambda0 # Beam waist size
  r = sqrt(y^2 + z^2)
  phi = atan2(y,z)

  m = 1
  p = 0
  Cpm = 1 #sqrt(p!/(m+p)!)
  n = 2.4e26
end:constant


begin:laser
  boundary = x_min
  lambda = lambda0
  amp = a0 / lambda0 * 2 * pi * me*c^2/qe # E0
  phase = m*phi
  profile = Cpm * (sqrt(2)*r/w_0)^m * gauss(r,0,w_0)
  t_profile = gauss(time, 0.0, 20*T_l)
end:laser


begin:species
   name = electron
   charge = -1.0
   mass = 1.0
   number_density = if((-45*lambda0 lt x) and (x lt 50*lambda0), n, 0.0)
   temperature_ev = 1e3
   nparticles_per_cell = 4
end:species


begin:species
   name = helium
   charge = 2.0
   mass = 7296.3
   number_density = 0.5*number_density(electron)
   temperature_ev = 30
   nparticles_per_cell = 2
end:species


begin:output
  name = normal

  # Number of timesteps between output dumps
  dt_snapshot = 15 * femto

  # Properties on grid
  grid = always
  ex = always
  ey = always
  ez = always
  poynting_flux = always

  # Properties at particle positions
  particle_grid = always
  px = always
  py = always
  pz = always

  number_density = always + species
  average_particle_energy = always + species
  particle_energy_flux = always + species
end:output

\end{verbatim}

\end{document}
